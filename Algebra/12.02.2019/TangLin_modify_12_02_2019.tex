\documentclass[12pt]{article}

\usepackage{amssymb,amsmath,amsthm}

\newtheoremstyle{neosn}{0.5\topsep}{0.5\topsep}{\rm}{}{\sc}{.}{ }{\thmname{#1}\thmnumber{ #2}\thmnote{ {\mdseries#3}}}
\theoremstyle{neosn}
\newtheorem{problem}{Problem}

\begin{document}
    \begin{center}
        {\bf Correction of Homework on 12.02.2019}\\
        Tanglin
    \end{center}

    \problem{Correct.}

    \problem{Is group $\mathbb{Z}_2 \times D_3$ and $D_6$ isomorphic?}\\
    Solution:\\
    We can describe group $D_3$ and $D_6$ in following forms:
    \[
        D_3 = \{\langle \varphi_{D_3},d_{D_3} \rangle| \varphi_{D_3}^3=d_{D_3}^2=e_{D_3},d_{D_3}\varphi_{D_3}d_{D_3}=\varphi_{D_3}^{-1}\}
        \]
    \[
        D_6 = \{\langle \varphi_{D_6},d_{D_6} \rangle| \varphi_{D_6}^6=d_{D_6}^2=e_{D_6},d_{D_6}\varphi_{D_6}d_{D_6}=\varphi_{D_6}^{-1}\}
        \]
    So we can write $\mathbb{Z}_2\times D_3$ as form:
    \[
        \mathbb{Z}_2\times D_3 = \{(a,b)|a \in \mathbb{Z}_2,b\in D_3\}
        \]
    Where $(a_1,b_1)(a_2,b_2)=(a_1a_2,b_1b_2)$ and $a_1,a_2  \in \mathbb{Z}_2,b_1b_2 \in D_3$.\\
    We will consider a map:
    \begin{align*}
        f:\mathbb{Z}_2 \times D_3 &\to D_6\\
        (z,e_{D_3}) &\longmapsto \varphi_{D_6}^3\\
        (e_Z,d_{D_3}) &\longmapsto d_{D_6}\\
        (e_Z,\varphi_{D_3}) &\longmapsto \varphi_{D_6}^{4}\\
        f(r_1r_2) &= f(r_1)f(r_2)\\
    \end{align*}
    We can prove that $f$ is a function.\\
    $\forall r_1,r_2 \in \mathbb{Z}_2\times D_3$\\
    $\exists l_1,m_1,n_1$ and $l_2,m_2,n_2 \in \mathbb{Z}$, s.t. $r_1 = (z^{l_1},\varphi_{D_3}^{m_1}d_{D_3}^{n_1}),r_2 = (z^{l_2},\varphi_{D_3}^{m_2}d_{D_3}^{n_2})$\\
    If $r_1 = r_2$, we can get:
    \begin{align}
        l_1 &\equiv l_2 (\text{mod 2})\\
        m_1 &\equiv m_2 (\text{mod 3})\\
        n_1 &\equiv n_2 (\text{mod 2})
    \end{align}
    Then
    \begin{align*}
        f(r_1) &= f((z^{l_1},\varphi_{D_3}^{m_1}d_{D_3}^{n_1}))\\
                &= f((z^{l_1},e_{D_3})(e_Z,\varphi_{D_3}^{m_1})(e_Z,d_{D_3}^{n_1}))\\
                &= f((z^{l_1},e_{D_3}))f((e_Z,\varphi_{D_3}^{m_1}))f((e_Z,d_{D_3}^{n_1}))\\
                &= f((z,e_{D_3}))^{l_1}f((e_Z,\varphi_{D_3}))^{m_1}f((e_Z,d_{D_3}))^{n_1}\\
                &= \varphi_{D_6}^{3l_1} \varphi_{D_6}^{4m_1}d_{D_6}^{n_1}\\
                &= \varphi_{D_6}^{3l_1+4m_1}d_{D_6}^{n_1}
    \end{align*}
    In the same way, we can get:
    $$f(r_2)=\varphi_{D_6}^{3l_2+4m_2}d_{D_6}^{n_2}$$\\  
    From (1),(2),(3), we have:
    \begin{align*}
        l_1 &= 2k_1+l_2\\
        m_1 &=3k_2+m_2\\
        n_1 &=2k_3+n_2
    \end{align*}
    So
    \begin{align*}
        3l_1 &= 3\cdot 2k_1+3l_2=6k_1+3l_2\\
        4m_1 &= 4\cdot 3k_2+4m_2=6\cdot 2k_2 + 4m_2\\
        n_1 &= 2k_3+n_2
    \end{align*}
    And Then
    \begin{align*}
        3l_1 &\equiv 3l_2 (\text{mod 6})\\
        4m_1 &\equiv 4m_2 (\text{mod 6})\\
        n_1 &\equiv n_2 (\text{mod 2})
    \end{align*}
    So we reach our purpose:
    \begin{align*}
        3l_1+4m_1 &\equiv 3l_2+4m_2 (\text{mod 6})\\
        n_1 &\equiv n_2 (\text{mod 2})
    \end{align*}
    Based on this system, we can assert:
    $$\varphi_{D_6}^{3l_1+4m_1}d_{D_6}^{n_1} = \varphi_{D_6}^{3l_2+4m_2}d_{D_6}^{n_2}$$
    and
    $$f(r_1)=f(r_2)$$
    So $f$ is really a function.\\

    Then we consider another function $h$:
    \begin{align*}
        h:D_6 &\to \mathbb{Z}_2 \times D_3\\
            d_{D_6} &\longmapsto (e_Z,d_{D_3})\\
            \varphi_{D_6} &\longmapsto (z,\varphi_{D_3})\\
            h(s_1s_2)&=h(s_1)h(s_2)
    \end{align*}
    According to the same way, we can prove that $h$ is also a function.\\
    Now we can see $f\circ h$:\\
    $\forall g \in D_6,\exists m_1,n_1 \in \mathbb{Z}$ s.t. $g=\varphi_{D_6}^{m_1}d_{D_6}^{n_1}$\\
    \begin{align*}
        f\circ h(g) &= f(h(\varphi_{D_6}^{m_1}d_{D_6}^{n_1}))\\
                    &= f(h(\varphi_{D_6}^{m_1})h(d_{D_6}^{n_1})))\\
                    &= f((z,\varphi_{D_3})^{m_1}(e_Z,d_{D_3})^{n_1})\\
                    &= f((z,\varphi_{D_3}))^{m_1}f((e_Z,d_{D_3}))^{n_1}\\
                    &= \varphi_{D_6}^{7m_1}d_{D_6}^{n_1}\\
                    &= \varphi_{D_6}^{m_1}d_{D_6}^{n_1}\\
                    &= g
    \end{align*}
    $\therefore f\circ g = \varepsilon_{D_6}$
    On contrary, we will see $h \circ f$:
    $\forall g \in \mathbb{Z}_2\times D_3,\exists l_1,m_1,n_1 \in \mathbb{Z}$ s.t. $g= (z^{l_1},\varphi_{D_3}^{m_1}d_{D_3}^{n_1})$\\
    \begin{align*}
        h \circ f(g) &= h(f((z^{l_1},\varphi_{D_3}^{m_1}d_{D_3}^{n_1})))\\
                    &= h(f((z^{l_1},e_{D_3}))f((e_Z,\varphi_{D_3}^{m_1}))f((e_Z,d_{D_3}^{n_1})))\\
                    &= h(f((z,e_{D_3}))^{l_1}f((e_Z,\varphi_{D_3}))^{m_1}f((e_Z,d_{D_3}))^{n_1})\\
                    &= h(\varphi_{D_6}^{3l_1}\varphi_{D_6}^{4m_1}d_{D_6}^{n_1})\\
                    &=h(\varphi_{D_6})^{3l_1+4m_1}h(d_{D_6})^{n_1}\\
                    &=(z^{3l_1+4m_1},\varphi_{D_3}^{3l_1+4m_1})(e_Z,d_{D_3}^{n_1})\\
                    &=(z^{3l_1}z^{4m_1},\varphi_{D_3}^{3l_1}\varphi_{D_3}^{4m_1}d_{D_3}^{n_1})\\
                    &=(z^{l_1},\varphi_{D_3}^{m_1}d_{D_3}^{n_1})\\
                    &=g
    \end{align*}
    $\therefore h \circ f = \varepsilon_{\mathbb{Z}_2\times D_3}$\\
    Therefore we can get consequence, functions $f$ and $h$ are inverse functions of each other.\\
    So, $f$ is an isomorphism from $\mathbb{Z}_2\times D_3$ onto $D_6$.\\
    $\therefore \mathbb{Z}_2\times D_3 \cong D_6 $\\
    The proof has been completed.\\

    \problem{Correct.}
\end{document}