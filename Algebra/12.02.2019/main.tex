\documentclass[12pt]{article} % Используем стандартный класс документа "статья"

\usepackage[utf8x]{inputenc} % Используем кодировку UTF8, полезно для корректного поиска по готововму pfd-документу
\usepackage[english,russian]{babel} % Языковые пакеты
\usepackage{amssymb,amsmath,amsthm} % Пакеты математических символов

\newtheoremstyle{neosn}{0.5\topsep}{0.5\topsep}{\rm}{}{\sc}{.}{ }{\thmname{#1}\thmnumber{ #2}\thmnote{ {\mdseries#3}}}
\theoremstyle{neosn} % Стиль наших собственных окружений
\newtheorem{problem}{Задача} % Окружение для задач с автоматической нумерацией, аналогично можно задавать свои откружения для теорем, лемм, следствий и т. д.
\renewcommand{\proofname}{Доказательство}
% Красивые обозначения часто встречающихся групп
\newcommand{\SL}{\,\mathrm{SL}\,} 
\newcommand{\GL}{\,\mathrm{GL}\,}
\renewcommand{\S}{\,\mathrm{S}\,}
\newcommand{\A}{\,\mathrm{A}\,}
\newcommand{\D}{\,\mathrm{D}\,}
\newcommand{\V}{\,\mathrm{V}\,}
% Ажурные буквы для целых, действительных и комплексных чисел
\newcommand{\Z}{\,\mathbb{Z}\,} 
\newcommand{\R}{\,\mathbb{R}\,}
\renewcommand{\C}{\,\mathbb{C}\,}
% Обозначения некоторых часто встречающихся понятий
\newcommand{\dett}{\,\mathrm{det}\,} 
\newcommand{\charr}{\,\mathrm{char}\,}
\renewcommand{\Im}{\,\mathrm{Im}\,}
\newcommand{\diag}{\,\mathrm{diag}\,}
\newcommand{\Ker}{\,\mathrm{Ker}\,}
\newcommand{\rk}{\,\mathrm{rk}\,}
%Вы можете определить и использовать свои команды.

\begin{document}
\begin{center}
	{\bf \Large Домашняя работа от 12.02.2019}
\end{center}

\begin{center}
	Тан Линь
\end{center}

\begin{problem}
Здесь находится решение первой задачи о поиске центра и коммутанта группы, порожденной элементами $a, b$ со следующими соотношениями: $a^3=b^4=e, bab^{-1}=a^2$.
\end{problem}

Решение:
$$
G = \langle a,b \rangle, \quad O(a) = 3,\quad O(b) = 4,\quad bab^{-1}=a^2
$$
По определению
$$
G=\{a^{i_1}b^{j_1}a^{i_2}b^{j_2}\cdots a^{i_r}b^{j_r}|0 \leqslant i_1,\cdots i_r < 3;0 \leqslant j_1, \cdots j_r < 4, r = 1,2,\cdots\}
$$

$be=eb=a^3b=ba^3=b$,так как $bab^{-1}=a^2$,
и получим нижние формулы
\begin{gather*}
	bab^{-1} = a^2 \Rightarrow ba=a^2b \\
	ba^2 = ba\cdot a = a^2b\cdot a = a^2\cdot ba = a^2\cdot a^2b = ab \Rightarrow ba^2b^{-1}=a \\
	b^2a=b\cdot ba =b\cdot a^2b=ba^2\cdot b=ab\cdot b=ab^2 \\
	b^3a=b\cdot b^2a=b\cdot ab^2=ba\cdot b^2=a^2b\cdot b^2=a^2b^3 \\
	b^4a=ab^4=a
\end{gather*}
Поэтому произвольное $b^ma^n$ можно представиться в виде $a^{n^{*}}b^{m^{*}}$,потому что,
$$
b^ma^n=b^ma\cdot a^{n-1}=a^{b_1}b^{m_1}a^{n-1}=a^{n_2}b^{m_2}a^{n-2}=\cdots =a^{n^*}b^{m^*}
$$
Для произвольного элементы из $G$,имеет вид
\begin{align*}
	a^{i_1}b^{j_1}a^{i_2}b^{j_2}\cdots a^{i_r}b^{j_r}   &= a^{i_1}a^{i_1'}b^{j_1'}b^{j_2}a^{i_3}b^{j_3}\cdots a^{i_r}b^{j_r} \\
									&= a^{i_1+i_1'}b^{j_1'+j_2}a^{i_3}b^{j_3}\cdots a_{i_r}b^{j_r} \\
									&= \cdots \\
									&=a^{i^*}b^{j^*}
\end{align*}

И группа $G$ имеет вид
$$
G = \{ a^rb^s|0\leqslant r<3,0\leqslant s < 4\}
$$
$\forall a^{r_1}b^{s_1},a^{r_2}b^{s_2} \in G $,если $a^{r_1}b^{s_1}=a^{r_2}b^{s_2}$,то $a^{r_1-r_2}=b^{s_2-s_1}$,и имеет следующие
случая

\begin{enumerate}
	\item $r_1-r_2 \equiv 0(mod \ \  3)$\\
	$b^{s_2-s_1}=a^0=e \Rightarrow s_2-s_1 \equiv 0 (mod \ \ 4)$

	\item $r_1-r_2 \equiv 1(mod \ \  3)$ \\
		$a = b^{s_2-s_1}$ \\
		$\because O(a)=3,\quad O(b^{s_2-s_1}) \neq 3$\\
		$\therefore r_1-r_2 \not\equiv 1(mod \ \ 3)$

	\item $r_1-r_2 \equiv 2(mod \ \  3)$\\
		$a^2=b^{s_2-s_1}$\\
		$\because O(a)=3,\quad O(b^{s_2-s_1}) \neq 3$\\
		$\therefore r_1-r_2 \not\equiv 2 (mod \ \ 3)$
	
\end{enumerate}
Только первой случай рациональный,поэтому для произвольных,\\$a^{r_1}b^{s_1}=a^{r_2}b^{s_2}$,\\можем говорить что,
$r_1 \equiv r_2 (mod \ \ 3)$ и $s_1 \equiv s_2 (mod \ \ 4)$,поэтому $a^{r_1}=a^{r_2}$ и $b^{s_1}=b^{s_2}$,то 
$$
G = \{ e,a,a^2,b,b^2,b^3,ab,ab^2,ab^3,a^2b,a^2b^2,a^2b^3\}
$$
$|G|=12$\\
Теперь мы искаем элементы центра группы $G$,

\begin{enumerate}
	\item $e \in Z(G)$
	\item \begin{align*}
		ab = ba^2 \neq ba &\Rightarrow a,b \notin Z(G) \\
		ba^2=ab\neq a^2b &\Rightarrow a^2 \notin Z(G) \\
		aab=a^2b, \quad aba = aa^2b=b \Rightarrow \quad a^2b \neq b &\Rightarrow ab \notin Z(G)\\
		aa^2b=b, \quad a^2ba=a^2a^2b=ab \Rightarrow \quad b \neq ab &\Rightarrow a^2b \notin Z(G)\\
	\end{align*}
	\item \begin{align*}
		ba^2 = ab^2,\quad b^2a^2=ab^2a=a^2b^2 \Rightarrow b^2a^rb^s=a^rb^2b^s=a^rb^sb^2 &\Rightarrow b^2 \in Z(G)\\
		bab^2=a^2bb^2=a^2b^3,\quad ab^2b=ab^3 \Rightarrow a^2b^3 \neq ab^3 &\Rightarrow ab^2 \notin Z(G)\\
		ba^2b^2=abb^2=ab^3, \quad a^2b^2b=a^2b^3 \Rightarrow ab^3\neq a^2b^3 &\Rightarrow a^2b^2 \notin Z(G)
	\end{align*}
	\item \begin{align*}
		ab^3\neq a^2b^3=b^3a &\Rightarrow b^3 \notin Z(G)\\
		bab^3=a^2bb^3=a^2, \quad ab^3b=a \Rightarrow a \neq a^2 &\Rightarrow ab^3 \notin Z(G)\\
		ba^2b^3=abb^3=a,\quad a^2b^3b=a^2 \Rightarrow a \neq a^2 &\Rightarrow a^2b^3 \notin Z(G)
	\end{align*}
\end{enumerate}
И нашли центр группы $G$, $Z(G)=\{e,b^2\}$.\\

Мы уже знали,что $bab^{-1}=a^2$ и $ba^2b^{-1}=a$,теперь искаем коммутант группы $G$
\begin{align*}
	[a^{r_1}b^{s_1},a^{r_2}b^{s_2}] &= a^{r_1}b^{s_1}a^{r_2}b^{s_2}(a^{r_1}b^{s_1})^{-1}(a^{r_2}b^{s_2})^{-1}\\
									&= a^{r_1}b^{s_1}a^{r_2}b^{s_2}b^{-s_1}a^{-r_1}b^{-s_2}a^{-r_2}\\
									&= a^{r_1}b^{s_1-1}ba^{r_2}b^{-1}b^{s_2+1}b^{-s_1}a^{-r_1}b^{-s_2}a^{-r_2}\\
									&= a^{r_1}b^{s_1-s_1}a^{r_2^*}b^{s_2+s_1}b^{-s_1}a^{-r_1}b^{-s_2}a^{-r_2}\\
									&= a^{r_1+r_2^*}b^{s_2}a^{-r_1}b^{-s_2}a^{-r_2}\\
									&= a^{r_1+r_2^*}a^{r_1^*}a^{-r_2}\\
									&= a^*
\end{align*}

$\therefore G'=\{[g_1,g_2]|g_1,g_2\in G\} < \langle a \rangle$\\
$\because [a,b] = aba^{-1}b^{-1}=aa=a^2$\\
$\therefore G=\langle a \rangle = \{e,a,a^2\}$

То ответ:$Z(G)=\{e,b^2\},\quad G'= \{e,a,a^2\}$

\newpage
\begin{problem}
Изоморрфны ли группы$\Z_2\times \D_3\rightarrow\D_6$?
\end{problem}	

Решение:\\
Изоморрфны группы $\Z_2\times \D_3$ и $\D_6$.\\
Рассмотрим,пусть $\varphi = (1\  2\  3\  4\  5\  6)$
\begin{align*}
	\varphi^2&=(1\  3\  5)(2\  4\  6)\\
	(\varphi^2)^2=\varphi^4&=(1\  5\  3)(2\  6\  4)
\end{align*}
и пусть
\begin{align*}
	d_1 &= (2\  6)(3\  5)\\
	d_2 &= (1\  3)(6\  4)\\
	d_3 &= (1\  5)(2\  4)
\end{align*}

то для подгруппы $N = \{e,\varphi^2,\varphi^4,d_1,d_2,d_3\}$,очевидно $N \cong D_3$
$\because|N|=6$\\
$\therefore [D_6\colon N]=\frac{12}{6}=2 \Rightarrow N \lhd D_6$
\\
Рассмотрим\\
$K = \{e,\varphi^3\} \qquad \qquad \qquad \varphi^3=(1 \ 4)(2\ 5)(3 \ 6)$\\
$\forall \tau \in D_6$
\begin{align*}
	\tau \varphi^3 \tau^{-1} &= \tau \varphi \tau^{-1} \tau \varphi \tau^{-1} \tau \varphi \tau^{-1}
							&= (\tau \varphi \tau^{-1})^{3}
							&=(\tau(1)\ \tau(2)\ \tau(3)\ \tau(4)\ \tau(5)\ \tau(6) )^3
\end{align*}
$\sigma=(\tau(1)\ \tau(2)\ \tau(3)\ \tau(4)\ \tau(5)\ \tau(6)) \in D_6 \Rightarrow \exists n,\sigma^n \in D_6$\\
$\therefore \tau \varphi^3 \tau{-1}=(\sigma^3)^n\in K$\\
$\therefore K \lhd D_6$\\
$\because N \cap K = \{e\} \Rightarrow K \times N = D_6$\\
$\because N \cong D_3,\quad K \cong \mathbb{Z}_2$\\
$\therefore D_6 = K \times N \cong \mathbb{Z}_2 \times D_3$\\
Доказано.

\newpage
\begin{problem}
	Пусть $N \lhd G$.Обязательно ли существует такая подгруппа $K$ в группе $G$,что $G=N \leftthreetimes K$?
\end{problem}	
Решение:\\
Рассмотрим $G=\mathbb{Z}_4$\\
$\{0,2\} \lhd G$\\
Если $\exists K, G=N \leftthreetimes K \Rightarrow |K|=2 \Rightarrow K=N$\\
$\therefore K \cap N = \{0,2\} \neq \{0\}$\\
то не всех такая подгруппа сушествует.


\end{document}