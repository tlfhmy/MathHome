\documentclass[12pt]{article}

\usepackage{amssymb,amsmath,amsthm,mathrsfs}

\newtheoremstyle{neosn}{0.5\topsep}{0.5\topsep}{\rm}{}{\sc}{.}{ }{\thmname{#1}\thmnumber{ #2}\thmnote{ {\mdseries#3}}}
\theoremstyle{neosn}
\newtheorem{problem}{Problem}

\begin{document}
    \begin{center}
        {\bf Correction of Homework on 05.03.2019}\\
        Tanglin
    \end{center}
    
    \problem{Study all sunsets of some set with operations symmetric difference and intersection.
    Is it a ring? If it is, then is it commutative? Does it include identity element? Which elements are
    reversible, zero divisor, idempotent?}\\
    Solution:\\
    Assume that set $S$ is an abitrary non-empty set. And let:
    $$ \mathscr{M}(A):=\{D|D\subseteq A\} $$
    First of all, we will verify some operations of symmetric difference.\\
    Definition of symmetric difference:
    $$ R \bigtriangleup S:= \{x|(x\in R \& x \notin S) \lor (x\in S \& x\notin R)\} $$
    \begin{enumerate}
        \item $R \bigtriangleup S = S \bigtriangleup R$\\
            \begin{align*}
                R \bigtriangleup S &= \{x|(x\in R \& x \notin S) \lor (x\in S \& x\notin R)\}\\
                                    &= \{x|(x\in S \& x\notin R) \lor (x\in R \& x \notin S)\}\\
                                    & = S \bigtriangleup R
            \end{align*}

        \item $(R \bigtriangleup S) \bigtriangleup T = R \bigtriangleup (S \bigtriangleup T)$\\
            \begin{align*}
                (R \bigtriangleup S) \bigtriangleup T &= \{x|(x\in R \& x \notin S) \lor (x\in S \& x\notin R)\} \bigtriangleup T\\
                    &= \{x|((x\in R \& x \notin S) \lor (x\in S \& x\notin R)) \& x \notin T \\
                    &\lor x \in T \& (\lnot ((x\in R \& x \notin S) \lor (x\in S \& x\notin R))) \}\\
                    &= \{x|(x\in R \& x \notin S \& x\notin T) \lor (x\in S \& x\notin R \& x\notin T)\\
                    &\lor(x\in T \& (x\notin R \lor x\in S)\& (x\notin S \lor x\in R ))\}\\
                    &= \{x|(x\in R \& x \notin S \& x\notin T) \lor (x\in S \& x\notin R \& x\notin T)\\
                    & \lor (x\in T \& (x\notin R \& x\notin S \lor x\in R \& x\in S))\}\\
                    &= \{x|(x\in R \& x \notin S \& x\notin T) \lor (x\in S \& x\notin R \& x\notin T)\\
                    & \lor (x\in T \& x\notin R \& x\notin S) \lor (x\in T \& x\in R \& x\in S)\}\\
                    &= \{x|(x\in R \& x \notin S \& x\notin T) \lor (x\notin R \& x\in S \& x\notin T)\\
                    & \lor (x\notin R \& x\notin S \& x\in T) \lor (x\in R \& x\in S \& x\in T)\}\\
                    &= \{x|(x\in T \& x \notin S \& x\notin R) \lor (x\notin T \& x\in S \& x\notin R)\\
                    & \lor (x\notin T \& x\notin S \& x\in R) \lor (x\in T \& x\in S \& x\in R)\}\\
                    &= (T \bigtriangleup S) \bigtriangleup R\\
                    &= (S \bigtriangleup T) \bigtriangleup R\\
                    &= R \bigtriangleup (S \bigtriangleup T)
            \end{align*}

        \item $R\cap (S \bigtriangleup T) = (R \cap S) \bigtriangleup (R \cap T)$\\
            For $R\cap (S \bigtriangleup T)$, we have:
            \begin{align*}
                R\cap (S \bigtriangleup T) &= \{x|x\in R \& (x\in S \& x\notin T \lor x\in T \& x\notin S)\}\\
                    &= \{x|x\in R \& x\in S \& x\notin T \lor x\in R \& x\in T \& x\notin S\}
            \end{align*}
            For $(R \cap S) \bigtriangleup (R \cap T)$, we have:
            \begin{align*}
                (R \cap S) \bigtriangleup (R \cap T) &= \{(x\in R \& x\in S) \& (\lnot (x\in R \& x\in T))\\
                        &\lor (x\in R \& x\in T) \& (\lnot (x\in R \& x\in S))\}\\
                        &= \{x|x\in R \& x\in S \& (x\notin R \lor x\notin T) \\
                        &\lor x\in R \& x\in T \& (x\notin R \lor x\notin S)\}\\
                        &= \{x|x\in R \& x\in S \& x\notin T \lor x\in R \& x\in T \& x\notin S\}
            \end{align*}
            So we can assert that $R\cap (S \bigtriangleup T) = (R \cap S) \bigtriangleup (R \cap T)$.
    \end{enumerate}

    Now we will prove that set $\mathscr{M}(A)$ with operations symmetric difference and intersection is a ring.\\
    \begin{enumerate}
        \item $(\mathscr{M}(A),\bigtriangleup)$ is Abelian group.\\
            \begin{enumerate}
                \item Associativity:\\
                    $\because \forall R,S,T \in \mathscr{M}(A), (R\bigtriangleup S)\bigtriangleup T =  R\bigtriangleup (S\bigtriangleup T)$\\
                \item Commutativity:\\
                    $\because \forall R,S \in \mathscr{M}(A), R\bigtriangleup S = S \bigtriangleup R$\\
                \item Closure:\\
                    $\forall R,S \in \mathscr{M}(A) \Rightarrow R \subseteq A, S \subseteq A \Rightarrow R\bigtriangleup S \subseteq A
                    \Rightarrow R\bigtriangleup S \in \mathscr{M}(A)$
                \item Zero element exists:
                    Consider empty set in $\mathscr{M}(A)$\\
                    $\forall R \in \mathscr{M}(A), R \bigtriangleup \emptyset = \emptyset \bigtriangleup R = R$\\
                We have proved that it is a group.    
            \end{enumerate}

        \item Satisfy the law of distribution.\\
           From nature 3 of operation symmetric difference, we can get it.
        
        \item  Closure of operation intersection.
            $\forall R,S \subset A \Rightarrow R \cap S \subset A$
    \end{enumerate}

    We have proved that $(\mathscr{M}(A),\bigtriangleup,\cap)$ is a ring.

    
\end{document}