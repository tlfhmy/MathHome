\documentclass[a4paper,14pt]{article}
\linespread{1.2}
\usepackage{amsthm,amsmath,amssymb}
\begin{document}
    \begin{center}
        Tang Lin\\
        \today
    \end{center}

    Problem: Please classify all groups which consists of 18 elements.\\
    Solve:\\
    $$ 18 = 2\cdot 3^2 $$
    From The Third Theorem of Syllow, we have:
    \begin{align*}
        N_2 \equiv 1 (mod \ 2) \ \  N_2|9 &\Rightarrow N_2 = 1,3,9\\
        N_3 \equiv 1 (mod \ 3) \ \  N_3|2 &\Rightarrow N_3 = 1
    \end{align*}
    So we have 2 cases:
    \begin{enumerate}

        \begin{item}{$N_2 = 1, N_3 = 1$}
            In this situation,there is only one 2-Syllow subgroup and one 3-Syllow subgroup, so
            $G \cong S_2 \oplus S_3$, where $|S_2|=2,|S_3|=9$.\\
            There will be 2 situations:
            \begin{enumerate}
                \begin{item}
                    $$G \cong \mathbb{Z}_2 \oplus \mathbb{Z}_9 \cong \mathbb{Z}_{18} $$
                \end{item}
                \begin{item}
                    $$G \cong \mathbb{Z}_2 \oplus \mathbb{Z}_3 \oplus \mathbb{Z}_3 \cong \mathbb{Z}_3 \oplus \mathbb{Z}_6 $$
                \end{item}
            \end{enumerate}
        \end{item}

        \begin{item}{$N_2 \neq 1, N_3 = 1$}\\
            $S_3 \lhd G$\\
            So $G \cong S_3 \leftthreetimes K$,where $|K|=2$.\\
            \begin{enumerate}
                \begin{item}{$S_3 \cong \mathbb{Z}_9$}
                    We can get automorphism of $S_3 \cong \mathbb{Z}_9$: $\text{Aut}(\mathbb{Z}_9) = H \cong \text{Aut}(S_3)$,where $|H| = 6$.\\
                    So We can make a table of automorphism,we assume $S_3=\langle a \rangle$:
                    \begin{align*}
                        \varepsilon && e && a   && a^2 && a^3 && a^4 && a^5 && a^6 && a^7 && a^8\\
                        \sigma_1    && e && a^2 && a^4 && a^6 && a^8 && a   && a^3 && a^5 && a^7\\
                        \sigma_2    && e && a^4 && a^8 && a^3 && a^7 && a^2 && a^6 && a   && a^5\\
                        \sigma_3    && e && a^5 && a   && a^6 && a^2 && a^7 && a^3 && a^8 && a^4\\
                        \sigma_4    && e && a^7 && a^5 && a^3 && a   && a^8 && a^6 && a^4 && a^2\\
                        \sigma_5    && e && a^8 && a^7 && a^6 && a^5 && a^4 && a^3 && a^2 && a
                    \end{align*}
                    Where $\varepsilon,\sigma_1,\sigma_2,\sigma_3,\sigma_4,\sigma_5$ are elements of $\text{Aut}(S_3)$.\\
                    Without difficut we can get rank of each map of isomorphism:
                    \[
                        O(\sigma_1) = 6, O(\sigma_2)=3,O(\sigma_3)=6,O(\sigma_4)=3,O(\sigma_5)=2
                        \]
                    $\because$There is map with rank 6, so $\text{Aut}(S_3)\cong \mathbb{Z}_6$.\\
                    Now we assume $K=\langle b \rangle,|K|=2$, if we can find a homormorphism $\varphi:K \rightarrow \text{Aut}(S_3)$, then we can establish a group.\\
                    $\because$ In subgroup $K$, there are only elemnts with rank 2 and zero.\\
                    So we can construct a map:\\
                    \begin{align*}
                        \varphi:K &\rightarrow \text{Aut}(S_3)\\
                        e &\longmapsto \varepsilon\\
                        b &\longmapsto \sigma_5 \\
                    \end{align*}
                    So we can get a group with form:
                    \[
                        G\cong \langle a,b|a^9=b^2=e,bab=a^{-1} \rangle \cong D_9
                        \]
                \end{item}

                \begin{item}{$S_3\cong \mathbb{Z}_3 \oplus \mathbb{Z}_3$}\\
                    $\because|S_3|=9=3^2,\therefore S_3$ is Abelian group. We can write $S_3$ in form
                    \[
                        S_3=\langle a,b|a^3=b^3=1\rangle
                        \]
                    $\because S_3 \lhd G,\therefore cS_3c=S_3$,where $c \in K=\langle c \rangle,|K|=2$.\\
                    So $\exists i_1,j_1,i_2,j_2 \in \{0,1,2\}$,and 
                    \[ cac=a^{i_1}b^{j_1}\]
                    \[ cbc=a^{i_2}b^{j_2}\]
                    Therefore
                    \begin{align*}
                        cac &= a^{i_1}b^{j_1}\\
                        ccacc &= ca^{i_1}b^{j_1}c\\
                        a &= ca^{i_1}ccb^{j_1}c\\
                        a &= (cac)^{i_1}(cbc)^{j_1}\text{, becasue } ca^nc=(cac)(cac)\cdots(cac)=(cac)^n\\
                        a &= (a^{i_1}b^{j_1})^{i_1}(a^{i_2}b^{j_2})^{j_1}\\
                        \because S_3 \text{ is Abelian group, so}\\
                        a &= a^{i_1^2+i_2j_1}b^{i_1j_1+j_2j_1}
                    \end{align*}
                    Analogously, we can get
                    \begin{align*}
                        cbc &= a^{i_2}b^{j_2}\\
                        ccbcc &= ca^{i_2}b^{j_2}c\\
                        b &= ca^{i_2}ccb^{j_2}c\\
                        b &= (cac)^{i_2}(cbc)^{j_2}\\
                        b &= (a^{i_1}b^{j_1})^{i_2}(a^{i_2}b^{j_2})^{j_2}\\
                        b &= a^{i_1i_2+i_2j_2}b^{i_2j_1+j_2^2}
                    \end{align*}
                    Becase in $S_3$, $\langle a \rangle \cup \langle b \rangle = \emptyset$, so we can get a congruence equations system:
                    $$
                    \begin{cases}
                        i_1^2 + i_2j_1 \equiv 1 (\text{mod 3})\\
                        i_1j_1+ j_2j_1 \equiv 0 (\text{mod 3})\\
                        i_1i_2+ i_2j_2 \equiv 0 (\text{mod 3})\\
                        i_2j_1+ j_2^2  \equiv 1 (\text{mod 3})
                    \end{cases}
                    $$
                    To solve this equation, we can respectively let $i_1 \equiv 0,1,2 (\text{mod 3})$, and discuss them, but it is little complex.
                    \begin{enumerate}
                        \item $i_1 \equiv 0 (\text{mod 3})$\\
                            We can simplify this congruence equation system to:
                            $$
                            \begin{cases}
                            i_2j_1 \equiv 1 (\text{mod 3})\\
                            j_2j_1 \equiv 0 (\text{mod 3})\\
                            i_2j_2 \equiv 0 (\text{mod 3})\\
                            i_2j_1+ j_2^2  \equiv 1 (\text{mod 3})
                            \end{cases}
                            $$
                            It is not difficut to konw that $\j_2 \equiv 0 (\text{mod 3})$ and $j_1 \not \equiv 0 (\text{mod 3})$, and we can
                            more simplify it to just one equation $i_2j_1 \equiv 1 (\text{mod 3})$
                            And now we can get the first two solutions $(i_1,j_1,i_2,j_2) = (0,1,1,0) \text{or} (0,2,2,0)$.
                        \item $i_1 \equiv 1 (\text{mod 3})$\\
                            We can simplify this congruence equation system to:
                            $$
                            \begin{cases}
                                i_2j_1 \equiv 0 (\text{mod 3})\\
                                j_1(j_2+1) \equiv 0 (\text{mod 3})\\
                                i_2(j_2+1) \equiv 0 (\text{mod 3})\\
                                i_2j_1+ j_2^2  \equiv 1 (\text{mod 3})
                            \end{cases}
                            $$
                            If $j_1 \equiv 0 (\text{mod 3})$, then we can get system of two equations, 
                            $i_2(j_2+1) \equiv 0 (\text{mod 3})$ and
                            $i_2j_1+ j_2^2  \equiv 1 (\text{mod 3})$, so in this case we can get answers will have form $(i_1,j_1,i_2,j_2) = (1,0,*,*)$,
                            if $i_2 \equiv 0 (\text{mod 3})$, we will get answers $(1,0,0,1)$ and $(1,0,0,2)$, and if $i_2 \equiv 1 (\text{mod 3})$,
                            we will get answer $(1,0,1,2)$, if $i_2 \equiv 2 (\text{mod 3})$, get $(1,0,2,2)$.\\
                            If $j_1\not \equiv 0 (\text{mod 3})$, and must $i_2 \equiv 0 (\text{mod 3})$, we can get system with 2 equations
                            $j_1(j_2 + 1) \equiv (\text{mod 3})$ and $j_2^2 \equiv 1 (\text{mod 3})$, but if $j_2 \equiv 1 (\text{mod 3})$,we will
                            find that $j_1 \equiv 0 (\text{mod 3})$, so there will only be $j_2 \equiv 2 (\text{mod 3})$,and 
                            we can get solutions $(1,1,0,2),(1,2,0,2)$.
                        \item $i_1 \equiv 2 (\text{mod 3})$\\
                            We can simplify system to:
                            $$
                            \begin{cases}
                                i_2j_1 \equiv 0 (\text{mod 3})\\
                                j_1(j_2+2) \equiv 0 (\text{mod 3})\\
                                i_2(j_2+2) \equiv 0 (\text{mod 3})\\
                                i_2j_1+ j_2^2  \equiv 1 (\text{mod 3})
                            \end{cases}
                            $$
                            If $j_1 \equiv 0 (\text{mod 3})$, we can get system of 2 equations,$i_2(j_2+2) \equiv 0 (\text{mod 3})$ and
                            $j_2^2 \equiv 1 (\text{mod 3})$, in this case we will get answers with form $(2,0,*,*)$,if $j_2 \equiv 1 (\text{mod 3})$,
                            we will get answers $(2,0,0,1),(2,0,1,1) \text{ and } (2,0,2,1)$, if $j_2 \equiv 2 (\text{mod 3})$, we will get answer
                            $(2,0,0,2)$.
                            If $j_1 \not \equiv 0 (\text{mod 3})$, so there must be $i_2 \equiv 0 (\text{mod 3})$ and $j_2 + 2 \equiv 0 (\text{mod 3})$,
                            so answer are $(2,1,0,1)$ and $(2,2,0,1)$.
                    \end{enumerate}
                    So we have solved this congruence equations system.\\
                    
                    But there is a better method by using
                    computer program. This is Python code.
                    \begin{verbatim}
DM = []
for i in range(0,3):
    for j in range(0,3):
        for k in range(0,3):
            for l in range(0,3):
                DM.append((i,j,k,l))

#       i1^2  + i2*j1 = 1 (mod 3)
#       i1*j1 + j2*j1 = 0 (mod 3)
#       i1*i2 + i2*j2 = 0 (mod 3)
#       i2*j1 + j2^2  = 1 (mod 3)
#   There I use representations: i[0] ~ i1; i[1]~j1; i[2]~i2; i[3]~j2

Ans = []
for i in DM:
    if (i[0]**2 + i[2]*i[1]) % 3 == 1 and\
    (i[0]*i[1] + i[3]*i[1]) % 3 == 0 and\
    (i[0]*i[2] + i[2]*i[3]) % 3 == 0 and\
    (i[2]*i[1] + i[3]**2) % 3 == 1:
        Ans.append(i)

print(Ans)
                    \end{verbatim}
                    Then we will get answers:
                    $(i_1,j_1,i_2,j_2) \in \{(0, 1, 1, 0), (0, 2, 2, 0), (1, 0, 0, 1), (1, 0, 0, 2), (1, 0, 1, 2),\\
                     (1, 0, 2, 2), (1, 1, 0, 2), (1, 2, 0, 2), (2, 0, 0, 1), (2, 0, 0, 2), (2, 0, 1, 1), (2, 0, 2, 1),\\
                     (2, 1, 0, 1), (2, 2, 0, 1)\}$\\
                     For these solutions, we can construct conjugation relationship:
                     \begin{align*}
                         \text{1) } cac&=b      &   cbc&=a\\
                         \text{2) } cac&=b^2    &   cbc&=a^2\\
                         \text{3) } cac&=a      &   cbc&=b\\
                         \text{4) } cac&=a      &   cbc&=b^2\\
                         \text{5) } cac&=a      &   cbc&=ab^2\\
                         \text{6) } cac&=a      &   cbc&=a^2b^2\\
                         \text{7) } cac&=ab     &   cbc&=b^2\\
                         \text{8) } cac&=ab^2   &   cbc&=b^2\\
                         \text{9) } cac&=a^2    &   cbc&=b\\
                         \text{10)} cac&=a^2    &   cbc&=b^2\\
                         \text{11)} cac&=a^2    &   cbc&=ab\\
                         \text{12)} cac&=a^2    &   cbc&=a^2b\\
                         \text{13)} cac&=a^2b   &   cbc&=b\\
                         \text{14)} cac&=a^2b^2 &   cbc&=b
                     \end{align*}
                     But when we compare pairs [4), 9)],[5), 13)],[6), 14)],[7), 11)] and [8), 12)], we will discover that if we exchange status of a and b, and
                     they are the same. So, in fact we could just explore cases 1), 2), 3), 4), 5), 6), 7), 8), 10).\\
                     Now we will consider them:
                     \begin{enumerate}
                         \item $cac=b,cbc=a$
                            $$ G_1 = \langle a,b,c|a^3=b^3=c^2=e,ab=ba,cac=b,cbc=a \rangle $$
                        \item $cac=b^2,cbc=a^2$
                            $$ cac=b^2, cbc=a^2 \Rightarrow cac=b^2,cb^2c=(a^2)^2=a $$
                            $\therefore$ We can construct a map from $G_2$ onto $G_1$
                            \begin{align*}
                                \varphi:G_2 &\to G_1\\
                                c &\longmapsto c_*\\
                                a &\longmapsto a_*\\
                                b^2 &\longmapsto b_*\\
                                \varphi(g_1g_2) &= \varphi(g_1)\varphi(g_2)
                            \end{align*}
                            Where $g_* \in G_1$.\\
                            We can certify that $c,a$ and $b^2$ all are generators,and we can use $a$ and $b^2$ to generate group $S_3$, 
                            so next, we will prove that the map will be isomorphism. First of all we have to certify that it is a function:\\
                            Becasue $G$ includes 18 elements, and $|a|*|b|*|c|=3*3*2=18$, we will spontaneously consider that every element of $G$
                            can be represented as form $a^rb^sc^t$, now we can see:
                            \begin{align*}
                                1) \ \ \ a^{r}b^{s}c^t &= a^{r}b^{s}c^{t} = b^sa^rc^t\\
                                2) \ \ \ a^{r}cb^{s} &= a^{r}cb^scc\\
                                                    &= a^r(cbc)^sc\\
                                                    &= a^ra^{2s}c = a^{2s+r}c\\
                                3)\ \ \ ca^rb^s &= ca^rccb^scc\\
                                                &= b^{2r}a^{2s}c = cb^sa^r
                            \end{align*}
                            Therefore, it indeed exits with this form. For group $G_1$, analogously, we can get the same consequence.\\
                            Therefore for any elment $g$ in $G$, we can assume there will be $r,s,t \in \mathbb{Z}$, and $g = a^rb^sc^t$,so for any
                            $g_1 = a^{r_1}b^{s_1}c^{t_1}, g_2 = a^{r_2}b^{s_2}c^{t_2}$ and if $g_1 = g_2$, then we will get 
                            \begin{align*}
                                a^{r_1}b^{s_1}c^{t_1} &= a^{r_2}b^{s_2}c^{t_2}\\
                                b^{-s_2}a^{r_1-r_2}b^{s_1} &= c^{t_2-t_1}\\
                                a^{r_1-r_2}b^{s_1-s_2} &= c^{t_2-t_1}
                            \end{align*}
                            $\because S_3 \cap \langle c \rangle = \{e\}$, so $a^{r_1-r_2}b^{s_1-s_2} = c^{t_2-t_1}=e$, and infer that
                            $s_1 \equiv s_2 \text{(mod 3)}, r_1\equiv r_2 \text{(mod 3)}, t_1\equiv t_2 \text{(mod 2)}$\\
                            Now we can verify it is really function.
                            \begin{align*}
                                \varphi(g_1) &= \varphi(a^{r_1}b^{s_1}c^{t_1})\\
                                            &= \varphi(a^{r_1})\varphi(b^{s_1})\varphi(c^{t_1})\\
                                            &= (\varphi(a))^{r_1}(\varphi(b))^{s_1}(\varphi(c))^{t_1}\\
                                            &= {a_*}^{r_1}(\varphi(b^4))^{s_1}{c_*}^{t_1}\\
                                            &= {a_*}^{r_1}(\varphi(b^2))^{2s_1}c_*^{t_1}\\
                                            &= {a_*}^{r_1}{b_*}^{2s_1}c^{t_1}      
                            \end{align*}
                            and
                            $$\varphi(g_2) = {a_*}^{r_2}{b_*}^{2s_2}c^{t_2}$$
                            But as we know, they are equal; becasue $a^{r_1} = a^{r_2}, b^{2s_1}=b^{2s_2},c^{t_1}=c^{t_2}$. And
                            $\varphi(g_1) = \varphi(g_2)$. So map $\varphi$ is function,and according to our assumption, it is also
                            homormorphism; now we will prove that it is a bijection. To see other function $\psi$:
                            \begin{align*}
                                \psi:G_1 &\to G_2\\
                                c_* &\longmapsto c\\
                                a_* &\longmapsto a\\
                                b_* &\longmapsto b^2\\
                                \psi(g_{*1}g_{*2}) &= \varphi(g_{*1})\varphi(g_{2*})
                            \end{align*}
                            Where $g_{*1},g_{*2} \in G_1$.\\
                            For every element $g_*=a_*^rb_*sc_*^t$ in $G_1$, we have:
                            \begin{align*}
                                \varphi(\psi(a_*^rb_*^sc_*^t)) &= \varphi(\psi(a_*)^r\psi(b_*)^s\psi(c_*)^t)\\
                                                            &= \varphi(a^rb^{2s}c^t)\\
                                                            &= \varphi(a)^r\varphi(b)^{2s}\varphi(c)^t\\
                                                            &= a_*^r\varphi(b^4)^{2s}c_*^t\\
                                                            &= a_*^r\varphi(b^2)^{4s}c_*^t\\
                                                            &= a_*^rb_*^{4s}c_*^t\\
                                                            &= a_*^rb_*^sc_*^t\\
                                                            &= g_*
                            \end{align*}  
                            So $\varphi \circ \psi = e_{G_1}$. Analogously, for $\psi \circ \varphi$, if we assume that $g = a^rb^sc^t$
                            then we have:
                            \begin{align*}
                                \psi(\varphi(a^rb^sc^t)) &= \psi(\varphi(a)^r\varphi(b)^s\varphi(c)^t)\\
                                                        &= \psi(a_*^rb_*^{2s}c_*^t)\\
                                                        &= \psi(a_*)^r\psi(b_*)^{2s}\psi(c_*)^t\\
                                                        &= a^rb^{4s}c^t\\
                                                        &= a^rb^sc^t\\
                                                        &= g
                            \end{align*}
                            Therefore $\psi \circ \varphi = e_{G_2}$, so they are inverse mapping for each other, and $\varphi$ is acctually
                            bijection. We have proven that it is isomorphism.The only difference between $G_2$ and $G_1$ is
                            that we exchanged the symols of $b$ and $b^2$. So $G_2 \cong G_1$.
                        \item $cac=a, cbc=b$\\
                            In this situation, we will get
                            \[ca=ac,cb=bc,ab=ba \]
                            So $G_3$ will be an Abelian group, because$\forall a^{r_i}b^{s_i}c^{t_i} \in G_3,i = 1,2$,\\
                            \begin{align*}
                                a^{r_1}b^{w_1}c^{t_1} \cdot a^{r_2}b^{s_2}c^{t_2} &= a^{r_1+r_2}b^{s_1+s_2}c^{t_1+t_2}\\
                                                    &= a^{r_2}b^{s_2}c^{t_2} \cdot a^{r_1}b^{s_1}c^{t_1}
                            \end{align*}
                            But $N_2 \neq 1$,so $G_3$ is impossible Abelian group, therefore it leads to contradiction. And $G_3$ does not exist.
                        \item $cac=a,cbc=b^2$
                            $$cac=a,cbc=b^2 \Rightarrow cabc=caccbc=ab^2,cab^2c=ab$$
                            And now construct map from $G_4$ onto $G_1$:
                            \begin{align*}
                                \varphi:G_4 &\to G_1\\
                                c &\longmapsto c_*\\
                                ab &\longmapsto a_*\\
                                ab^2 &\longmapsto b_*\\
                                \varphi(g_1g_2) &= \varphi(g_1)\varphi(g_2)
                            \end{align*}
                            In the same way as we used for proof in previous situation, we can get analogous result.
                            We can change symols $a$ to $ab$ and $b$ to $ab^2$, then we will get $G_4$ from $G_1$. So $G_4 \cong G_1$.
                        \item $cac=a,cbc=ab^2$
                            $$cbc=ab^2 \Rightarrow cbc=ab^2,cab^2c=b$$
                            For map:
                            \begin{align*}
                                \varphi:G_5 &\to G_1\\
                                c &\longmapsto c\\
                                ab^2 &\longmapsto a\\
                                b &\longmapsto b\\
                                \varphi(g_1g_2) &= \varphi(g_1)\varphi(g_2)
                            \end{align*}
                            In the same way we used,we can prove that $\varphi$ is isomorphism $G_5 \cong G_1$.
                        \item $cac=a,cbc=a^2b^2$
                            $$cac=a,cbc=a^2b^2 \Rightarrow cbc = a^2b^2, ca^2b^2c=b$$
                            And consider this map:
                            \begin{align*}
                                \varphi:G_6 &\to G_1\\
                                c &\longmapsto c_*\\
                                a^2b^2 &\longmapsto a_*\\
                                b &\longmapsto b_*\\
                                \varphi(g_1g_2) &= \varphi(g_1)\varphi(g_2)
                            \end{align*}
                            Similarly, we can prove that $\varphi$ is isomorphism from $G_6$ onto $G_1$, we need just exchange status of $a$ and $a^2b^2$.
                            So $G_6 \cong G_1$.
                        \item $cac=ab, cbc=b^2$
                            $$cac=ab, cbc=b^2 \Rightarrow cac=ab, cabc=a$$
                            And map:
                            \begin{align*}
                                \varphi:G_7 &\to G_1\\
                                c &\longmapsto c_*\\
                                a &\longmapsto a_*\\
                                ab &\longmapsto b_*\\
                                \varphi(g_1g_2) &= \varphi(g_1)\varphi(g_2)
                            \end{align*}
                            Similarly, we can prove that $\varphi$ is isomorphism from $G_6$ onto $G_1$, we need just exchange status of $b$ and $ab$.
                            So $G_7 \cong G_1$.
                        \item $cac=ab^2, cbc=b^2$
                            $cac=ab^2, cbc=b^2 \Rightarrow cac = ab^2, cab^2c=a$
                            And map:
                            \begin{align*}
                                \varphi:G_8 &\to G_1\\
                                c &\longmapsto c_*\\
                                a &\longmapsto a_*\\
                                ab^2 &\longmapsto b_*\\
                                \varphi(g_1g_2) &= \varphi(g_1)\varphi(g_2)
                            \end{align*}
                            Similarly, we can prove that $\varphi$ is isomorphism from $G_6$ onto $G_1$, we need just exchange status of $b$ and $ab^2$.
                            So $G_8 \cong G_1$.
                        \item $cac=a^2, cbc=b^2$\\
                            In this situation, spontaneously, we will think of the same way as we used for previous situations. But soon we will find that it
                            becomes useless.\\
                            We can see 3 cases of them:
                            \begin{enumerate}
                                \item $cac = a^2,ca^2c=a$\\
                                    If we make map like:
                                    \begin{align*}
                                        \varphi:G_9 &\to G_1\\
                                        c &\longmapsto c_*\\
                                        a &\longmapsto a_*\\
                                        a^2 &\longmapsto b_*\\
                                        \varphi(g_1g_2) &= \varphi(g_1)\varphi(g_2)
                                    \end{align*}
                                    we will find that $a_* = \varphi(a^4) = \varphi(a^2)^2 = b_*^2$, but we know that $\langle a_* \rangle \cap 
                                    \langle b_* \rangle = \{e_*\}$, so it is impossible. And this $\varphi$ is not isomorphism.
                                \item $cbc = b^2, cb^2c=b$\\
                                    Similarly, if we construct map like:
                                    \begin{align*}
                                        \varphi:G_9 &\to G_1\\
                                        c &\longmapsto c_*\\
                                        b^2 &\longmapsto a_*\\
                                        b &\longmapsto b_*\\
                                        \varphi(g_1g_2) &= \varphi(g_1)\varphi(g_2)
                                    \end{align*}
                                    we will find that $b_* = \varphi(b^4) = \varphi(b^2)^2 = a_*^2$, but we know that $\langle a_* \rangle \cap 
                                    \langle b_* \rangle = \{e_*\}$, so it is impossible. And this $\varphi$ is not isomorphism.
                                \item $cabc = caccbc = a^2b^2, ca^2b^2c=ab$\\
                                    Analogously, if map like:
                                    \begin{align*}
                                        \varphi:G_9 &\to G_1\\
                                        c &\longmapsto c_*\\
                                        ab &\longmapsto a_*\\
                                        a^2b^2 &\longmapsto b_*\\
                                        \varphi(g_1g_2) &= \varphi(g_1)\varphi(g_2)
                                    \end{align*}
                                    we wil find that $e_*=\varphi(e)=\varphi(a^3b^3)=\varphi(aba^2b^2)=\varphi(ab)\varphi(a^2b^2)=a_*b_*$, this 
                                    is also impossible, and $\varphi$ is not isomorphism.
                            \end{enumerate}
                            Now we have enough reasons to consider that there is not isomorphism between $G_9$ and $G_1$.\\
                            We will prove our conjecture. Assume that there is an isomorphism $\varphi$\\
                            Because $O(c_*)=2, O(c)=2$,and they are the only element with rank 2 in respective group, so $\varphi(c)=c_*$.\\
                            So $\exists a^rb^sc^t: \varphi(a^rb^sc^t)=a_*$,where $r,s \in \{0,1,2\},t \in \{0,1\}$,there will be 2 cases:
                            \begin{enumerate}
                                \item $t = 0$\\
                                    In this case, $\varphi(a^rb^s)=a_*$
                                    \begin{align*}
                                        \varphi(a^rb^s)&=a_*\\
                                        \varphi(ca^rb^sc) &= \varphi(c)a_*\varphi(c)\\
                                        \varphi(ca^rc)\varphi(cb^sc) &= c_*a_*c_*\\
                                        \varphi(cac)^r\varphi(cbc)^s &= b_*\\
                                        \varphi(a^2)^r\varphi(b^2)^s &= b_*\\
                                        \varphi((a^rb^s)^2) &= b_*\\
                                        \varphi(a^rb^s)^2 &= b_*\\
                                        a_*^2 &= b_*
                                    \end{align*}
                                    It is impossible, because in group $G_1$, $\langle a_* \rangle \cap \langle b_* \rangle = \{e_*\}$, and
                                    it causes contradiction. \newpage
                                \item $t = 1$\\
                                    In this case, $\varphi(a^rb^sc)=a_*$
                                    \begin{align*}
                                        \varphi(a^rb^sc)&=a_*\\
                                        \varphi(a^rb^s)\varphi(c) &= a_*\\
                                        \varphi(ca^rb^rc)c_* &= \varphi(c)a_*\varphi(c)\\
                                        \varphi(ca^rc)\varphi(cb^sc) &= c_*a_*\\
                                        \varphi(a^rb^s)^2 &= c_*a_* \\
                                        \varphi(a^rb^s)\varphi(c)\varphi(c)\varphi(a^rb^s)\varphi(c)\varphi(c) &= c_*a_*\\
                                        \varphi(a^rb^sc)\varphi(c)\varphi(a^rb^sc)\varphi(c) &= c_*a_*\\
                                        a_*c_*a_*c_* &= c_*a_*\\
                                        a_*b_* &= c_*a_*\\
                                        a_*b_*a_*^{-1} = c_*\\
                                        b_* = c_*
                                    \end{align*}
                                    It is no impossible, and causes contradiction.
                            \end{enumerate}
                            And it has been proven that there is not isomorphism between $G_9$ and $G_1$. So $G_9 \not \cong G_1$.\\
                            So
                            $$
                            G_9 = \langle a,b,c|a^3=b^3=c^2=e,ab=ba,cac=a^2,cbc=b^2\rangle
                            $$
                            is another type of group which consists of 18 elemnts.
                     \end{enumerate}
                \end{item}
                And we had find all the types of group which consists of 18 elements.
            \end{enumerate}
        \end{item}
    \end{enumerate}
    They are 
    \begin{align*}
        1)& \mathbb{Z}_{18}\\
        2)& \mathbb{Z}_3\oplus \mathbb{Z}_6\\
        3)& D_9\\
        4)& \langle a,b,c|a^3=b^3=c^2=e,ab=ba,cac=b,cbc=a\rangle\\
        5)& \langle a,b,c|a^3=b^3=c^2=e,ab=ba,cac=a^2,cbc=b^2\rangle
    \end{align*}
    We have completed classification of groups which consists of 18 elemnts.
\end{document}