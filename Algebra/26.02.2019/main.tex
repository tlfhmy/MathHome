\documentclass[a4paper,12pt]{article}

\usepackage[utf8x]{inputenc}
\usepackage[english,russian]{babel}
\usepackage{amssymb,amsmath,amsthm}

\newtheoremstyle{neosn}{0.5\topsep}{0.5\topsep}{\rm}{}{\sc}{.}{ }{\thmname{#1}\thmnumber{ #2}\thmnote{ {\mdseries#3}}}
\theoremstyle{neosn}
\newtheorem{problem}{Задача}
\renewcommand{\proofname}{Доказательство}

\begin{document}
\begin{center}
    {\bf \Large Домашнее задание от 19.02.2019}\\
    Тан Линь
\end{center}

\begin{problem}
    Найдите коммутант группы диэдра $D_n$. Сколько в нем элементов?
    (Отдельно рассмотрите случаи четного и нечетного $n$). Разрешима ли группа $D_n$?
\end{problem}
Решение:\\
\[
    D_n := \langle a,b \rangle\ \ O(a)=n,\ O(b)=2,\ bab=a^{-1},\ \forall k \not \equiv 0(mod \ \ n),b \neq a^k
    \]
\[
    D_n=\{a^{r_1}b^{s_1}\cdots a^{r_k}b^{s_k}|0 \leqslant r_i < n,s_k \in \{0,1\} \}
    \]
\begin{align*}
    bab = a^{-1} &\Rightarrow ba = a^{-1}b = a^{n-1}b \\
    ba^2  =ba \cdot a = a^{n-1}ba = (a^{n-1})^2b=a^{n-2}b &\Rightarrow ba^2 = a^{n-2}b\\
    \vdots \\
    ba^{n-1}=ba^{n-2}a=a^2ba=a^2a^{n-1}b=ab &\Rightarrow ba^{n-1}=ab
\end{align*}
То   у нас есть 
$$
    ba^r=a^{n-r}b=a^{-r}b
$$
Поэтому для любого элемента из группы $D_n$:
\[
    a^{r_1}b^{s_1}\cdots a^{r_k}b^{s_k} = a^{r'}b^{s'}
    \]
где $0 \leqslant r' < n$ и $s' \in \{0,1\}$\\
Теперь докажем что, для $a^{r_1}b^{s_1}=a^{r_2}b^{s_2}$, то $r_1 \equiv r_2 (mod \ \ n)$ и $s_1 \equiv s_2 (mod\ \ 2)$, так 
как $a^{r_1 - r_2} = b^{s_2-s_1}$ и $\forall k \not \equiv 0(mod \ \ n),b \neq a^k$.

Теперь рассмотрим коммутант группы $D_n$:
\[
    D_n'=[D_n,D_n] = \{[g_1,g_2]|g_1 \in G,g_2 \in G\}
    \]
\begin{enumerate}
    \begin{item}
        $[a^r,a^s]=e$
    \end{item}
    \begin{item}
        $[a_r,b]=a^rba^{-r}b=a^ra^r=a^{2r}$
    \end{item}
    \begin{item}
        $[a^rb,a^s]=a^rba^sba^{-r}a^{-s}=a^ra^{-s}a^{-r}a^{-s}=a^{-2s}$
    \end{item}
    \begin{item}
        $[a^rb,b]=a^rbbba^{-r}b=a^ra^r=a^{2r}$
    \end{item}
\end{enumerate}
то $D_n' \subseteq \langle a^2 \rangle$, но$[a,b]=aba^{-1}b=a^2$,то $\langle a^2 \rangle \subseteq D_n'$, то $D_n'=\langle a^2 \rangle$

Если
\begin{enumerate}
    \begin{item}{$n \equiv 0 (mod \ \ 2)$}
        \[D_n'=\langle a^2 \rangle,\quad |D_n'|=\frac{n}{2}\]            
    \end{item}
    \begin{item}{$n \equiv 1 (mod \ \ 2)$}
        \[D_n'=\langle a^2 \rangle = \langle a \rangle,\quad |D_n'|=n\]
    \end{item}
\end{enumerate}
То $D_n \rhd D_n'=\langle a^2 \rangle \rhd D_n''=[D_n',D_n']=\{e\}$, поэтому группа $D_n$ разрешима.
\\
\\
\\
\\
\\
\begin{problem}
    Докажите, что любая группа из\\
    а) 20,\\
    б) 100 \\
    элементов разрешима.
\end{problem}
Решение:\\
а) По теореме Силова 3\\
\begin{align*}
    N_2 \equiv 1 (mod \ \ 2), \ \ N_2|5 &\Rightarrow N_2 = 1 \text{ или } N_2 = 5 \\
    N_5 \equiv 1(mod \ \ 5),\ \ N_5|4 &\Rightarrow N_5=1
\end{align*}
То $\exists H \lhd G$, где $|H| = 5$ и $|G| = 20$\\
$\therefore G \rhd H \cong \mathbb{Z}_5 \rhd [H,H] = \{e\}$\\
$\therefore G$ разрешима.\\
\\
б)\\
$$
N_5 \equiv 1 (mod \ \ 5),\ \ N_5 | 4 \Rightarrow N_5 = 1
$$
То $G \rhd H$, $\exists |H|=5^2=25$\\
$\because 5 \text{ простое число}$\\
$\therefore H$ абелева.\\
$\therefore G \rhd H \rhd [H,H] = {e}$
$\therefore G(|G|=100)$ разрешима.
\end{document}
